\documentclass[12pt]{article}

\usepackage[shortlabels]{enumitem}
\usepackage{geometry}
 \geometry{
 a4paper,
 total={180mm,257mm},
 left=10mm,
 top=10mm,
 }
\usepackage{amsmath}
\usepackage{amssymb}
\usepackage{amsfonts}
\usepackage{mathtools}
\usepackage{tcolorbox}
\usepackage{xparse}
% \usepackage{hyperref}

\DeclarePairedDelimiter\abs{\lvert}{\rvert}%
\newcommand{\limit}[2]{ \underset{#1\rightarrow #2}{\lim}\, }
\newcommand{\convg}[1]{ \overset{#1}{\rightarrow} }
\newcommand{\var}[1]{ \text{Var}(#1)}
\newcommand{\cov}[2]{ \text{Cov}(#1, #2)}
\newcommand*{\expec}[2][]{\underset{\scriptscriptstyle{#1}}{\mathbb{E}}[#2]}
\newcommand{\Exp}[1]{ \text{Exponential}(#1)}
\newcommand{\Normal}[2]{ \mathcal{N}(#1, #2)}
\newcommand{\Beta}[2]{ \text{Beta}(#1, #2)}
\newcommand{\Pois}[1]{ \text{Poisson}(#1)}
\newcommand{\Poisf}[2]{ #2^{#1} \frac{e^{-#2}}{#1!}}
\newcommand{\distiid}{ \overset{iid}{\sim}}
\newcommand{\prompt}[1]{\begin{tcolorbox}\textbf{\underline{Prompt}}: \textit{#1}\end{tcolorbox}}
\newcommand*{\samp}[2][n]{#2_{1:#1}}
\newcommand{\partiald}[2]{\genfrac{}{}{0pt}{1}{\partial#1}{\partial #2}}
\newcommand{\secpartiald}[2]{\genfrac{}{}{0pt}{1}{\partial#1^2}{\partial #2^2}}
\renewcommand{\vec}[1]{\boldsymbol{#1}}
\newcommand{\tp}{^\intercal}
\newcommand{\loglike}[1]{\mathcal{L}(#1)}
\newcommand{\fishinf}[1]{\mathcal{I}(#1)}


\renewcommand{\thesubsection}{\thesection.\alph{subsection}}

\begin{document}
\title{Homework 3 Responses}
\author{Chris Crabtree}
\date{Oct. 28, 2021}

{
\let\clearpage\relax
\maketitle

}

\section{}
I spent a lot of time on questions 3,4, and 5 and could not finish this.

      
\stepcounter{section}
\subsection{}
    % !TEX root = ../*.tex

\item Prove that if $Y \sim  \Pois{\lambda} \rightarrow \expec{Y}  =  \lambda \text{ and }  \var{Y} = \lambda$

    The first part can be shown directly by the definition of $\expec{Y}$:
    \begin{align*}
        \expec{Y} &= \sum_{-\infty}^\infty y\Poisf{y}{\lambda} \tag*{(by definition)}\\
                  &= \sum_{y=0}^\infty y \Poisf{y}{\lambda} \tag*{(Poisson region of interest)}\\
                  &= e^{-\lambda}\sum_{y=0}^\infty \frac{y \lambda^y}{y!} \tag*{(Algebra)}\\
                  &= e^{-\lambda}\lambda \sum_{y=1}^\infty \frac{\lambda^{y-1}}{(y-1)!} \tag*{(Algebra)}\\
                  &= e^{-\lambda}\lambda \sum_{k=0}^\infty \frac{\lambda^k}{k!} \tag*{(Substitute \(k=y-1\))}\\
                  &= e^{-\lambda}\lambda e^\lambda  \tag*{(Taylor expansion for \(e^x)\)}\\
                  &= \lambda 
    \end{align*}

    Similarly, for $\var{Y}$ we have:
    \begin{align*}
        \var{Y} &= \expec{Y^2} - \expec{Y}^2   \tag*{(by definition)}\\
                &= \sum_{y=0}^\infty y^2\Poisf{y}{\lambda} - \lambda^2\\
                &= e^{-\lambda}\lambda \sum_{y=1}^\infty y\frac{\lambda^{y-1}}{(y-1)!}- \lambda^2 \tag*{(Above steps)}\\
                &= e^{-\lambda}\lambda \sum_{y=1}^\infty ((y-1)+1)\frac{\lambda^{y-1}}{(y-1)!}- \lambda^2 \\
                &= e^{-\lambda}\lambda \sum_{y=1}^\infty \Big((y-1)\frac{\lambda^{y-1}}{(y-1)!} + \frac{\lambda^{y-1}}{(y-1)!}\Big)- \lambda^2 \\
                &= e^{-\lambda}\lambda \Big(\sum_{y=1}^\infty (y-1)\frac{\lambda^{y-1}}{(y-1)!} + \sum_{y=1}^\infty \frac{\lambda^{y-1}}{(y-1)!}\Big)- \lambda^2 \\
                &= e^{-\lambda}\lambda \Big(\sum_{y=1}^\infty (y-1)\frac{\lambda^{y-1}}{(y-1)!} +e^\lambda \Big)- \lambda^2 \tag*{(Same as above)}\\
                &= e^{-\lambda}\lambda \Big(\lambda \sum_{y=2}^\infty \frac{\lambda^{y-2}}{(y-2)!} +e^\lambda \Big)- \lambda^2 \\
                &= e^{-\lambda}\lambda \Big(\lambda e^\lambda +e^\lambda \Big)- \lambda^2 \\
                &= \lambda^2 + \lambda - \lambda^2 \\
                &=  \lambda  \\
    \end{align*}


\subsection{}
    % !TEX root = ../*.tex

\item Decide which estimator is better $\lambda$, $\bar{Y}_n$ or $s^2_n$

    We have shown in class that both $\bar{Y}_n$ and $s^2_n$ are unbiased, so here we will calculate the variance.
    \begin{enumerate}

        \item \var{$\bar{Y}_n$}:
            \begin{align*}
                \var{\bar{Y}_n} &= \expec{\bar{Y}_n^2}- \expec{\bar{Y}_n}^2\\
                                &= \frac{1}{n^2}\expec{\sum_{i=1}^n Y_i\sum_{j=1}^n Y_j}- \lambda^2\\
                                &= \frac{1}{n^2}\sum_{i=1}^n\sum_{j=1}^n\expec{Y_i Y_j}- \lambda^2\\
                                &= \frac{1}{n^2}\Big(\sum_{i=j}\expec{Y_i Y_j}+ \sum_{i\neq j}\expec{Y_i Y_j}\Big)- \lambda^2\\
                                &= \frac{1}{n^2}\Big(\sum_{i=1}^n\expec{Y_i^2}+ \sum_{i\neq j}\expec{Y_i}\expec{ Y_j}\Big)- \lambda^2\tag*{(by independence)}\\
                                &= \frac{1}{n^2}\Big(\sum_{i=1}^n(\var{Y_i}+\expec{Y_i}^2)+ n(n-1)\lambda^2\Big)- \lambda^2\\
                                \tag*{($\expec{Y_i^2} =\var{Y_i} + \expec{Y_i}^2$, by def. of Var($\cdots$))}\\
                                &= \frac{1}{n^2}\Big(n(\lambda+\lambda^2)+ n(n-1)\lambda^2\Big)- \lambda^2\\
                                &= \frac{1}{n}\lambda+\frac{1}{n}\lambda^2+\lambda^2 - \frac{1}{n}\lambda^2- \lambda^2\\
                                &= \frac{1}{n}\lambda
            \end{align*}
        \item \var{$s^2_n$}:

            From slide 29 of lecture we know that: 
                \begin{equation*}
                    \var{s^2_n} = \frac{\mu_4}{n}
                \end{equation*}
            Online I found that $\mu_4$ for the Poisson distribution is $\lambda^4 + 7\lambda^3 + 6\lambda^2 + \lambda$ from taking the fourth derivative of the MGF of the Poisson evaluated at zero. 
            This gives
                \begin{equation*}
                    \var{s^2_n} = \frac{\lambda^4 + 7\lambda^3 + 6\lambda^2 + \lambda}{n} \geq \frac{1}{n}\lambda = \var{\bar{Y}_n}
                \end{equation*}
                Therefore I would choose $\bar{Y}_n$ as $\hat{\lambda}$.

    \end{enumerate}



\stepcounter{section}
\subsection{}
    Plots are given in figure \ref{fig:Reg_loc_only}. 
I used the log-likelihood function as my stopping criteria.
For the these non-stochastic experiments I stopped iterating when the improvement in the log-likelihood was less than .001.




\begin{figure}
     \centering
     \begin{subfigure}[b]{0.3\textwidth}
         \centering
         \includegraphics[width=\textwidth]{../code/regular_loc_only_plots/galaxies_hist_k_4.png}
         \caption{K=4}
         \label{fig:Reg_loc_only4}
     \end{subfigure}
     \hfill
     \begin{subfigure}[b]{0.3\textwidth}
         \centering
         \includegraphics[width=\textwidth]{../code/regular_loc_only_plots/galaxies_hist_k_6.png}
         \caption{K=6}
         \label{fig:Reg_loc_only6}
     \end{subfigure}
     \hfill
     \begin{subfigure}[b]{0.3\textwidth}
         \centering
         \includegraphics[width=\textwidth]{../code/regular_loc_only_plots/galaxies_hist_k_8.png}
         \caption{K=8}
         \label{fig:Reg_loc_only8}
     \end{subfigure}
     \begin{subfigure}[b]{0.3\textwidth}
         \centering
         \includegraphics[width=\textwidth]{../code/regular_loc_only_plots/galaxies_hist_k_11.png}
         \caption{K=11}
         \label{fig:Reg_loc_only11}
     \end{subfigure}
     \hfill
     \begin{subfigure}[b]{0.3\textwidth}
         \centering
         \includegraphics[width=\textwidth]{../code/regular_loc_only_plots/galaxies_hist_k_15.png}
         \caption{K=15}
         \label{fig:Reg_loc_only15}
     \end{subfigure}
     \hfill
     \begin{subfigure}[b]{0.3\textwidth}
         \centering
         \includegraphics[width=\textwidth]{../code/regular_loc_only_plots/galaxies_hist_k_20.png}
         \caption{K=20}
         \label{fig:Reg_loc_only20}
     \end{subfigure}
        \caption{Non-Stochastic 1-D Location Mixtures}
        \label{fig:Reg_loc_only}
\end{figure}

\subsection{}
    Tabulated AIC and BIC scored below:
\vspace{4mm}

\begin{tabular}{lllr}
\toprule
{} &     aic &     bic &  iters \\
k  &         &         &        \\
\midrule
4  &  433.4* &  455.1* &     88 \\
6  &   450.7 &   482.0 &     40 \\
8  &   449.4 &   490.4 &    267 \\
11 &   440.0 &   495.4 &    173 \\
15 &   448.6 &   523.2 &    179 \\
20 &   468.7 &   567.4 &    476 \\
\bottomrule
\end{tabular}

\vspace{4mm}
\noindent Best values of each are marked with a *.


\subsection{}
    Plots for the location-scale models are given in figure \ref{fig:Reg_loc_scale}.
For the location-scale models I limited the $\sigma^2$'s to have a minimum of .001.

\begin{figure}
     \centering
     \begin{subfigure}[b]{0.3\textwidth}
         \centering
         \includegraphics[width=\textwidth]{../code/regular_loc_scale_plots/galaxies_hist_k_3.png}
         \caption{K=3}
         \label{fig:Reg_loc_scale3}
     \end{subfigure}
     \hfill
     \begin{subfigure}[b]{0.3\textwidth}
         \centering
         \includegraphics[width=\textwidth]{../code/regular_loc_scale_plots/galaxies_hist_k_4.png}
         \caption{K=4}
         \label{fig:Reg_loc_scale4}
     \end{subfigure}
     \hfill
     \hfill
     \begin{subfigure}[b]{0.3\textwidth}
         \centering
         \includegraphics[width=\textwidth]{../code/regular_loc_scale_plots/galaxies_hist_k_5.png}
         \caption{K=5}
         \label{fig:Reg_loc_scale5}
     \end{subfigure}
     \hfill
     \begin{subfigure}[b]{0.3\textwidth}
         \centering
         \includegraphics[width=\textwidth]{../code/regular_loc_scale_plots/galaxies_hist_k_6.png}
         \caption{K=6}
         \label{fig:Reg_loc_scale6}
     \end{subfigure}
     \begin{subfigure}[b]{0.3\textwidth}
         \centering
         \includegraphics[width=\textwidth]{../code/regular_loc_scale_plots/galaxies_hist_k_7.png}
         \caption{K=7}
         \label{fig:Reg_loc_scale7}
     \end{subfigure}
     \hfill
     \begin{subfigure}[b]{0.3\textwidth}
         \centering
         \includegraphics[width=\textwidth]{../code/regular_loc_scale_plots/galaxies_hist_k_8.png}
         \caption{K=8}
         \label{fig:Reg_loc_scale8}
     \end{subfigure}
     % \hfill
     % \begin{subfigure}[b]{0.3\textwidth}
     %     \centering
     %     \includegraphics[width=\textwidth]{../code/regular_loc_scale_plots/galaxies_hist_k_9.png}
     %     \caption{K=}
     %     \label{fig:Reg_loc_scale}
     % \end{subfigure}
        \caption{Non-Stochastic 1-D Location-Scale Mixtures}
        \label{fig:Reg_loc_scale}
\end{figure}

\newpage
\subsection{}
    
Tabulated AIC and BIC scored below:
\vspace{4mm}

\begin{tabular}{lllr}
\toprule
{} &     aic &     bic &  iters \\
k &         &         &        \\
\midrule
3 &   424.4 &  446.0* &     64 \\
4 &   425.1 &   454.0 &     71 \\
5 &  410.1* &   446.2 &    123 \\
6 &   425.9 &   469.3 &    549 \\
7 &   410.5 &   461.1 &    522 \\
8 &   412.0 &   469.7 &    499 \\
\bottomrule
\end{tabular}

\vspace{4mm}
\noindent Best values of each are marked with a *.

\subsection{}
    It seemed that the location only models produced more reliable results. 
Allowing the variance parameters to update allowed the excess centroids to increase the log-likelihood arbitrarily by decreasing the variance to zero.
This is why I limited the minimum $\sigma^2$'s.

It also appeared that initialization played a large role in the goodness of fit. 
The resulting mixture pdf's could vary substantially on repeated runs.



\section{}
\prompt{Show that Jeffrey's priors satisfy the invariance principle}

In Jeffrey's view, a prior is non-informative if the prior chosen for $p_{\psi}(\psi)$ is proportional to the pdf that appears with the transformation $p_{\theta}(\psi=g(\theta)) = p_\psi(\psi)$.
The Jeffrey's prior for $p(\psi)$ would be $\big|I(\psi)\big|^{1/2}$ and the pdf of the transformation $p_{\theta}(\psi=g(\theta)) = p_\psi(\psi)$ is a well-known result if $g$ is monotone, i.e. $p_\psi(\psi)= p_\theta(g^{-1}(\psi))\big|\partiald{}{\psi}g^{-1}(\psi)\big|$.
Therefore, I need to show that $p_{\theta}(\psi=g(\theta)) \propto \big|I(\psi)\big|^{1/2}$.
Starting from the pdf of the transformation $p_{\theta}(\psi=g(\theta))$:
\begin{align*}
    p_{\theta}(\psi=g(\theta))&= p_\theta(g^{-1}(\psi))\big|\partiald{}{\psi}g^{-1}(\psi)\big|\\
                             &=\big|\secpartiald{}{\theta}\log(p_\theta(y\big|\theta=g^{-1}(\psi)))\big|^{1/2}\big|\partiald{}{\psi}g^{-1}(\psi)\big|\\
                              \Rightarrow\\
p_{\theta}(\psi=g(\theta))^2 &= \big|\secpartiald{}{\theta}\log(p_\theta(y\big|\theta=g^{-1}(\psi)))\big|\big|\partiald{}{\psi}g^{-1}(\psi)\big|^2\\
                             &= I(g^{-1}(\psi))\big|\partiald{}{\psi}g^{-1}(\psi)\big|^2\\
\end{align*}
From here I need to use a special property of Fisher information.
This property relates the information of paramater to the fisher information of a function of a parameter.
It states that if $\alpha$ and $\beta = h(\alpha)$ are scalar parameters and if $h$ is a continuously differentiable function, then the Fisher information of the two parameterizations are related by the following equation:
\begin{align*}
    I(\alpha) &= I(h(\alpha))(\frac{d }{d\alpha}h(\alpha))^2\\
\end{align*}
This is tricky for me to prove, but it seems to be a well known result.
Using this property and letting $h = g^{-1}$ I get:
\begin{align*}
     p_{\theta}(\psi=g(\theta))^2 &= I(g^{-1}(\psi))\big|\partiald{}{\psi}g^{-1}(\psi)\big|^2\\
                                &= I(\psi)^2\\
                              \Rightarrow\\
     p_{\theta}(\psi=g(\theta)) &= I(\psi)^{1/2}\\
\end{align*}
Which is the Jeffrey's prior for $p(\psi)$ that I needed to prove.
% there is a property of Fisher information that states that under a reparameterization $h: \alpha \rightarrow \beta$, 
%
% \begin{align*}
%                               &\propto \big|\secpartiald{}{\psi}\log(p_\psi(y\big|\psi))\big|^{1/2}
% \end{align*}



\section{}
I sadly did not have time to complete this.


Show that if $X \sim N(\mu, \sigma^2$, then $M_{X}(t) = e^{\mu t +\sigma^2 t^2/2}$.

I will show this directly from the definition:
\begin{align*}
    M_{X}(t) = \expec{e^{tX}} &= \int e^{tx}\frac{1}{\sigma\sqrt{2\pi}}e^{-(\frac{x-\mu}{\sqrt{2}\sigma})^2}dx\\
            &= \frac{1}{\sigma\sqrt{2\pi}}\int e^{tx-(\frac{x-\mu}{\sqrt{2}\sigma})^2}dx\\
            &= \frac{1}{\sigma\sqrt{2\pi}}\int e^{t(u\sigma+\mu)-(\frac{u}{\sqrt{2}})^2}\sigma du \tag*{let $u = \frac{x-\mu}{\sigma}$}\\
            &= \frac{e^{t\mu}}{\sqrt{2\pi}}\int e^{-\frac{u^2}{2}+t\sigma u} du \\
            &= \frac{e^{t\mu}}{\sqrt{2\pi}}\int e^{-\frac{1}{2}(u+t\sigma)^2+\frac{t^2\sigma^2}{2}} du \tag*{(completing the square)}\\
            &= e^{t\mu}e^{\frac{t^2\sigma^2}{2}}\int\frac{1}{\sqrt{2\pi}} e^{-\frac{1}{2}(u+t\sigma)^2} du \\
            &= e^{t\mu+\frac{t^2\sigma^2}{2}} \tag*{(Normal($u+t\sigma$, 1) integrates to 1)}\\
            \blacksquare
\end{align*}




% !TEX root = ../*.tex
Binomial problems
\begin{enumerate}
    \item
        % !TEX root = ../*.tex
Show that  for $R \sim Bin(\pi, m)$, $K_{Bin}(t)$ the cumulant generating function $\log(M_{Bin}(t))$ is $m\log(1-\pi +\pi e^t)$. 

We will start by deriving the MGF of the Binomial distribution, \\${M_{Bin}(t) = (1-\pi +\pi e^t)^m}$.
\begin{align*}
    M_{Bin}(t) &= \expec{e^{Rt}} = \sum_{x=0}^m e^{xt}\cdot P(R=x)\\ 
               &=\sum_{y=0}^m e^{xt} {m \choose x}\pi^x(1-\pi)^{m-x}\\
                 &=\sum_{y=0}^m  {m \choose x} (e^t \pi)^x(1-\pi)^{m-x}\\
\end{align*}
Recall from the Binomial Theorem that:
\begin{equation}
    \label{bin_thm}
    (z+y)^n = \sum_{k=0}^n{n \choose k}z^{k}y^{n-k} = \sum_{k=0}^n{n \choose k}z^{n-k}y^{k}
\end{equation}
By choosing $z := e^t \pi$, $y:= 1 - \pi$, $k:=x$, and $m:=n$, we have that:

\begin{align*}
    \sum_{y=0}^m  {m \choose x} (e^t \pi)^x(1-\pi)^{m-x} = (\pi e^t + 1-\pi)^m\\
\end{align*}
Thus, 
\begin{align*}
    K_{Bin}(t) = log(M_{bin}(t) =  m\log(\pi e^t + 1-\pi)\\
    \blacksquare
\end{align*}



    \item
        % !TEX root = ../*.tex
\label{q7_CGF}Show that $\limit{\pi}{0}\limit{m}{\infty} K_{Bin}(t)$ implies $m\pi \rightarrow \lambda$, for some constant $\lambda > 0$. 
\begin{align*}
   \limit{\pi}{0}\limit{m}{\infty} K_{Bin}(t) &=\limit{\pi}{0}\limit{m}{\infty}  m\log(\pi e^t + 1-\pi)= m\log( 1 +\pi e^t -\pi)\\
         &=\limit{\pi}{0}\limit{m}{\infty}\log\Big( (1 +\pi (e^t -1))^m\Big)
\end{align*}
Recall that $\limit{n}{\infty}(1+x/n)^n = e^x$.
This gives:
\begin{align*}
    \limit{m}{\infty}\limit{\pi}{0}\log\Big( (1 +\pi (e^t -1))^m\Big) &= \limit{m}{\infty}\limit{\pi}{0}\log\Big( (1 +\frac{m\pi (e^t -1)}{m})^m\Big)\\
                          &= \limit{m}{\infty}\limit{\pi}{0}\log\Big( (1 +\frac{\lambda (e^t -1)}{\lambda/\pi})^{(\lambda/\pi)}\Big) \tag*{(let $ \lambda = m\pi$)}\\
                          &=\limit{m}{\infty}\log(e^{\lambda(e^t-1)}) \\
                          &=\limit{m}{\infty}\lambda(e^t-1) \\
                          &=\lambda(e^t-1) \\
    \blacksquare
\end{align*}



    \item
        % !TEX root = ../*.tex
Show that $\limit{\pi}{0}\limit{m}{\infty} K_{Bin}(t)\convg{D} Pois(\lambda)$

We showed in part \ref{q7_CGF} that $\limit{\pi}{0}\limit{m}{\infty} K_{Bin}(t) = \lambda(e^t-1)$.
The MGF of the Binomial is just $e^{K_{Bin}(t)}$. The same proof given above holds for $\limit{\pi}{0}\limit{m}{\infty} e^{K_{Bin}(t)}$.
This means that in the limits stated, the MGF of the binomial converges to $e^{\lambda(e^t-1)}$. 
This is exactly the MGF of the Poisson distribution. 
Thus, by the Uniquess Theorem of MGFs:
\begin{equation*}
    \limit{\pi}{0}\limit{m}{\infty} P(R=r) \convg{D} \frac{\lambda^{r}}{r!}e^{-\lambda} = \text{Poisson}(m\pi=\lambda)
\end{equation*}


    \item
        \input{contents/q7_d.tex}
\end{enumerate}


Normal problems
\begin{enumerate}
\item If $Z \sim $ Normal(0, 1), derive the density of $Y = Z^2$.
    We can derive the density of $Y$ using the CDF technique for random variable transformations.
    \begin{align*}
        F_Y(y) &= P(Y \leq y)\\
               &= P(X^2 \leq y)\\
               &= P( |X| \leq \sqrt{y})\\
               &= P(-\sqrt{y} \leq X \leq \sqrt{y})\\
               &= F_X(\sqrt{y})-F_X(-\sqrt{y})\\
    \end{align*}
    Now we just need to differentiate the CDF to get the PDF.
    \begin{align*}
        F_Y'(y) &= \frac{d}{dy}({F_X(\sqrt{y})-F_X(-\sqrt{y})})\\
                &= \frac{d}{dy}{F_X(\sqrt{y})-\frac{d}{dy}F_X(-\sqrt{y})}\\
                &= \frac{f_X(\sqrt{y})}{2\sqrt{y}}+\frac{f_X(-\sqrt{y})}{2\sqrt{y}}\\
                &= \frac{1}{2\sqrt{y2\pi}}e^{-\frac{1}{2}(\sqrt{y})^2}+\frac{1}{2\sqrt{y2\pi}}e^{-\frac{1}{2}(-\sqrt{y})^2}\\
                &= \frac{1}{2\sqrt{y2\pi}}e^{-\frac{y}{2}}+\frac{1}{2\sqrt{y2\pi}}e^{-\frac{y}{2}}\\
                &= \frac{1}{\sqrt{y2\pi}}e^{-\frac{y}{2}}\\
    \end{align*}

    This is a chi-squared distribution

\item Show that $Y$ is uncorrelated with $Z$.

    To do this we will look at $\cov{Y}{Z}$ which is the numerator of the correlation calculation.

    \begin{align*}
        \cov{Y}{Z} &= \expec{YZ} - \expec{Y}\expec{Z}\\
                   &= \expec{YZ} - 0\tag*{($\expec{Z}= 0$)}\\
                   &= \expec{Z^3} - 0\tag*{($\expec{Z}= 0$)}\\
                   &= 0 - 0\tag*{(odd moments of the normal are 0)}\\
    \end{align*}
    Since $\cov{Y}{Z}$ the correlation is also zero.
\end{enumerate}


\end{document}

